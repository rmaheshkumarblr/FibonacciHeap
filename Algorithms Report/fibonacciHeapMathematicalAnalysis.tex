We use potential method to analyze the performance of Fibonacci Heap Operations. For a Fibonacci Heap H, let us indicate the number of trees in the root list of H by $t(H)$ and the number of marked nodes in H by $m(H)$. \\ The potential of Fibonacci Heap H is defined by \\
\begin{center}
	$\Phi{(H)} \ = \ t(H) + 2m(H)$
\end{center}
Clearly,
\begin{itemize}
	\item The potential when there is no heaps is zero.
	\item The potential is non-negative at all subsequent times.
	\item The amortized cost upper bounds the actual cost
\end{itemize}

The Mathematical Analysis for the run time of the various operations are as follows:

\subsection{Inserting a node}
Initial potential:
\begin{center}
	$\Phi{(H)} \ = \ t(H) + 2m(H)$
\end{center}
After insertion of a node, let the Fibonacci heap be denoted by $H'$.
\begin{center}
$t(H') \ = \ t(H) + 1$ \\
$m(H') \ = \ m(H)$ \\
\end{center}
Increase in potential = New potential - Old potential
\begin{equation}
	\begin{split}
		\triangle{\Phi} &= [t(H')+2m(H')] - [t(H)+2m(H)]
		\\&= [t(H) + 1 + 2m(H)] - [t(H)+2m(H)]
		\\&= 1
	\end{split}
\end{equation}
Amortized cost = Actual cost + Increased potential
\begin{equation}
\begin{split}
Amortized \ cost &= O(1) + 1
\\&= O(1)
\end{split}
\end{equation}
\subsection{Union of Fibonacci Heaps}
Initial potential:
\begin{center}
	$\Phi{(H_1)} \ = \ t(H_1) + 2m(H_1)$ \\
	$\Phi{(H_2)} \ = \ t(H_2) + 2m(H_2)$
\end{center}
Let Fibonacci Heap tree formed by the union of $H_1$ and $H_2$ be denoted by $H'$
\begin{center}
$t(H') \ = \ t(H_1) + t(H_2)$ \\
$m(H') \ = \ m(H_1) + m(H_2)$ \\
$\Phi{(H')} \ = \ t(H') + 2m(H')$
\end{center}
Increase in potential is given by
\begin{equation}
	\begin{split}
		\triangle{\Phi} &= [t(H')+2m(H')] \\ 
		& - [ (t(H_1) + 2m(H_1)) + (t(H_2) + 2m(H_2)) ]
		\\&= [ (t(H_1) + 2m(H_1)) + (t(H_2) + 2m(H_2)) ] \\
		& - [ (t(H_1) + 2m(H_1)) + (t(H_2) + 2m(H_2)) ]
		\\&= 0
	\end{split}
\end{equation}
Amortized cost = Actual cost + Increased potential
\begin{equation}
\begin{split}
Amortized \ cost &= O(1) + 0
\\&= O(1)
\end{split}
\end{equation}
\subsection{Extracting the Minimum Node}
Initial potential:
\begin{center}
	$\Phi{(H)} \ = \ t(H) + 2m(H)$
\end{center}
The actual cost of extracting the minimum node can be accounted as follows: 
\begin{itemize}
	\item An $O(D(n))$ contribution comes from there being at most $D(n)$ children of the minimum node that are added to the root list and processed.\\
	\item The size of the root list on calling the consolidate function is at most $D(n)+t(H)-1$, since there were originally $t(H)$ trees, at most $D(n)$ of nodes can be the children to the node that is being extracted and one $(1)$ since that is the node that is being extracted. Thus the total actual work done in extracting the minimum node is $O(D(n)+t(H))$
\end{itemize}
The change in potential is given as follows:
\begin{itemize}
	\item Potential at most, post extracting the node is \\ $(D(n)+1 + 2m(H))$, since at most $D(n)+1$ roots remain and no nodes become marked during the operation.
\end{itemize}
Amortized cost = Actual cost + Increased potential
\begin{equation}
\begin{split}
Amortized \ cost &= O(D(n)+t(H)) \\
& + [ [(D(n)+1+2m(H)] - [t(H)+2m(H)] ]
\\&= O(D(n)) + O(t(H)) + D(n) + 1 -t(H)
\\&= O(D(n))
\end{split}
\end{equation}
\subsection{Decreasing a key}
Initial potential:
\begin{center}
	$\Phi{(H)} \ = \ t(H) + 2m(H)$
\end{center}
The actual cost for decreasing a key can be accounted as follows: 
\begin{itemize}
	\item It takes $O(1)$ time, plus the time to perform cascading cuts.
	\item Suppose the cascading cut is recursively called $c$ times and each call takes $O(1)$ time exclusive of recursive call, it takes $O(c)$ including all recursive call.
\end{itemize}
The change in potential is given as follows:
\begin{itemize}
	\item Each recursive call of cascading cut, except for the last one, cuts a marked node and clears the marked bit.
	\item There are hence, $(t(H) + c)$ trees
	\item There are at most $( m(H) - c + 2 )$ marked nodes
\end{itemize}
Amortized cost = Actual cost + Increased potential
\begin{equation}
	\begin{split}
		Amortized \ cost &= O(c) \\
		& + [ (t(H) + c) + 2(m(H) - c + 2) - [t(H)+2m(H)] ]
		\\&= O(c) - c + 4
		\\&= O(1)
	\end{split}
\end{equation}
\subsection{Deleting a node}
Deleting a node is equivalent to decreasing a key and then extracting the minimum.\\
Hence, 
\begin{equation}
	\begin{split}
		Amortized \ cost &= O(D(n)) + O(1)\\
		\\&= O(D(n))
	\end{split}
\end{equation}
\subsection{Bounding the maximum degree}
To show that the Extracting the minimum node operation and the Deleting a node operation takes $O(\lg{n})$ we need to prove that the upper bound of $D(n)$ on the degree of any node of a $n$-node Fibonacci heap is $O(\lg{n})$. \\ \\
Let x be any node in a Fibonacci Heap, and suppose the $degree[x] = k $. If $y_1, y_2, y_3,...,y_k$ denote the children of $x$ in the order in which they were linked to $x$, from the earliest to the latest. Then, $degree[y_1] \geq \ 0$ and $degree[y_i] \geq \ i - 2$ for $i=2,3,...,k$ because when $y_i$ was linked to $x$, all of $y_1,y_2,...,y_{i-1}$ were children of $x$, so we must have had $degree[x] \geq \ i-1$. Node $y_i$ is linked to $x_i$ only if their degree are the same and hence $degree[y_i] = degree[x] \geq i - 1$. Since then, $y_i$ has lost at most one child, since it would have been cut from x if it had lost two children. We conclude that $degree[y_i] \geq i-2$\\ \\
From the property of Fibonacci number: We have
\begin{center}
	$F_{k} = 0 \ if \ k=0$\\
	$F_{k} = 1 \ if \ k=1$\\
	$F_{k} = F_{k-1}+F_{k+2} \ if \ k \geq 2$\\
\end{center}
For all integers $k \geq 0$ indicate the smallest possible tree of degree k , we get number of nodes in the sequence of Fibonacci numbers using the Fibonacci property: $n \geq \Phi^{k}$, where $\Phi \approxeq 1.1618$ and is the golden ratio. $F_{k+2} \geq \Phi^{k}$\\ \\
For any node in a Fibonacci heap, let $k=degree[x]$. Then, we can show that the $size(x) \geq F_{k+2} \geq \Phi^{k}, \ \\where \ \Phi = \frac{(1+\sqrt{5})}{2} .$ \\ \\
Assuming the number of nodes as $n$.Taking base-$\Phi$ log on both the sides,we get $k \leq \log_{\Phi}{n}$.\\ The maximum degree $D(n)$ of any node is thus $O(\lg{n})$. 
\\ \\ \\ \\ \\ \\ \\ \\ \\ \\ \\ \\
\begin{center}
	\textbf{Time Complexity for Fibonacci Heap Operations}
 \begin{tabular}{c c c c||} 
 \hline
 Operation \& Time Complexity \\ [0.5ex] 
 \hline\hline
 Find Min & $O(1)$ \\ 
 \hline
 Insert & $O(1)$ \\
 \hline
 Union & $O(1)$ \\
 \hline
 Extract-Min & $O(\lg{n})*$ \\
 \hline
 Decrease-Key & $O(1)*$ \\
 \hline
 Delete & $O(\lg{n})*$ \\ [1ex] 
 \hline
\end{tabular}\\
$*$ indicates amortized
\end{center}









