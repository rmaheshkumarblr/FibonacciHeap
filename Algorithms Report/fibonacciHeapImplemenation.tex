For the implementation of the Fibonacci Heap, I created a FibonacciHeap class and a Node class.
The FibonacciHeap Node class has the following variables:
\begin{itemize}
	\item key: Indicates the value of the key(data)
	\item degree: The number of immediate child nodes
	\item parent: The parent of the node
	\item child: The child of the node
	\item left \& right: The siblings if any are pointed using these pointers (circular, doubly linked list)
	\item mark: Used for tracking history and restricting the amortized cost for extracting minimum and deleting a node to $O(\lg{n})$  
\end{itemize}
The FibonaaciHeap class has the following variables: 
\begin{itemize}
	\item rootList: Pointer to a node of the root list
	\item minNode: Node with the minimum key value
	\item totalNode: Total number of nodes within the Fibonacci Heap
\end{itemize}
The FibonacciHeap class has multiple functions defined, their uses are as follows:
\subsection{ insert (Insert a node to the Fibonacci Heap) }
\begin{itemize}
		\item We create a new node
		\item Make left and right pointer point to itself
		\item Call the mergeNewNodeWithRootList function
		\item If minNode was undefined or it has key greater than the new node, we update minNode to point to the new node
		\item We increase the totalNode count by 1 indicating that a new node was added to the Fibonacci Heap.
\end{itemize}

\subsection{ extractMin ( Extracting the Minimum Node ) }
	\begin{itemize}
		\item We check if the minNode is undefined or not
		\item If minNode does exist, we check if the node has child node
			\begin{itemize}
			\item If there exist a child Node, then we get the list of all the children using the iterateNodeDoubleLinkList function
			\item We add all the children to the root list using the mergeNewNodeWithRootList function
			\end{itemize}
		\item We remove the minimum node from the root list using the removeFromRootList function
		\item We update the minNode to a random node in the root list
		\item We call the consolidate function 
		\item We decrease the total nodes by 1 as the minimum node is removed from the Fibonacci Heap
	\end{itemize}
	
\subsection{consolidate ( Reduce the number of min-heap trees from the Fibonacci Heap )}
		\begin{itemize}
			\item We create an empty list of size ( total number of nodes in the Fibonacci Heap ) called A
			\item We get all the nodes in the root list using the  iterateNodeDoubleLinkList function
			\item For each node, we iterate and get it's degree
			\begin{itemize}
				\item If the list A, has any node corresponding to the degree
				\item We figure out which node has lower key value and add the larger key value node as a child to the lower key value node using the heapLink function
				\item We update the list A with index as degree of the node to empty
				\item We increment the degree of the node by 1
			\end{itemize}
			\item We update the minNode with the node which is minimum among the root list
		\end{itemize}
\subsection{decreaseKey ( Decreasing a key and updating the Fibonacci Heap)}
		\begin{itemize}
			\item If the min-heap property is violated
			\begin{itemize}
				\item We run the cut function
				\item We run the cascadingCut function
			\end{itemize}
			\item We update the minNode appropriately.
		\end{itemize}
\subsection{ cut ( If the child node is smaller than its parent after the decrease key operation, we cut the child and bring it back to the root list )}
		\begin{itemize}
			\item We call the removeFromChildList function
			\item We decrease the degree by 1 and call the mergeNewNodeWithRootList function to add the removed child to the root list
			\item We mark the parent to empty and unmark it when it is added to the root list
		\end{itemize}
\subsection{ cascadingCut ( Based on the history ( mark property ), we decide if we need to recursively go up the tree, post a cut operation )}
		\begin{itemize}
			\item If the node is not marked, we mark it
			\item If the node has been marked already, we recursively iterate up to the root or until a node is not marked by calling the cut and cascadingCut function.	
		\end{itemize}
\subsection{heapLink ( Remove a node from the rootList and add it as a child to another node of the rootList, maintaining the circular double linked list properties)}
		\begin{itemize}
			\item Call the removeFromRootList function for the node that needs to be removed from the root list.
			\item Call the mergeWithChild function to merge the node as a child of another node
			\item Increase the degree of the node to which the node is added as child, update hat parent pointer and update the mark to False
		\end{itemize}
\subsection{removeFromRootList ( Remove a node from the circular double linked root list )}
		\begin{itemize}
			\item If the pointer rootList is pointing to it, update the pointer to another node in the root list
			\item Update the pointer of the doubly linked list to remove the node
		\end{itemize}
\subsection{iterateNodeDoubleLinkList ( Iterating through the circular double linked list for a given node )}
		\begin{itemize}
			\item If the node is undefined, return none.
			\item If the node is defined, iterate through a particular direction of the doubly linked list till we reach back to the original node. 
			\item Return all the nodes that were traversed through.
		\end{itemize}
\subsection{mergeRootListWithExistingRootList ( Joining two circular doubly linked root list )}
		\begin{itemize}
			\item Update the pointers to join both the doubly linked lists of root list
		\end{itemize}
\subsection{mergeNewNodeWithRootList ( Adding a node to doubly linked root list )}
		\begin{itemize}
			\item If the rootList pointer is empty update the pointer to point to the node
			\item If the rootList pointer is not empty, add the node to the circular doubly linked root list
		\end{itemize}
\subsection{mergeWithChild ( Adding a node as the child of a given parent node )}
		\begin{itemize}
			\item If the parent has no child, just update the child pointer to point to the node
			\item If the parent has a child already, update the pointers to add the node to circular doubly linked list that contains all its children.
		\end{itemize}
\subsection{removeFromChildList ( Removing a node from its parent )}
		\begin{itemize}
			\item If it is the only child, update the parent's child pointer as empty
			\item If the parent had multiple child, update parent pointer with another child and the child's parent pointer with the parent node
			\item Update the pointers to remove the node from the circular doubly linked list
		\end{itemize}

